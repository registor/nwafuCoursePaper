% 西北农林科技大学科技类及IT类课程论文文档类(LaTeX模板)
\documentclass{nwafucoursepaper}

% 载入需要的宏包
\usepackage{etoolbox}

% =========浮动体增强宏包=========
\usepackage{floatrow}
\floatsetup[figure]{objectset=centering, margins=centering}

% % ========处理标题的宏包=========
% \usepackage[labelsep=quad]{caption}
% \usepackage{varioref}
% \usepackage{subfig}

% =========引号宏包=========
\usepackage{csquotes}

% 中文行距调整
\usepackage{zhlineskip}


%%% Local Variables:
%%% mode: latex
%%% TeX-master:"../main.tex"
%%% End:

% 进行必要的设置
% 重定义强调字体的代码,解决默认强调字体是italic,此时中文会用楷体代替,
% 在此设置为加粗,注意需要使用etoolbox宏包
\makeatletter
\let\origemph\emph
\newcommand*\emphfont{\normalfont\bfseries}
\DeclareTextFontCommand\@textemph{\emphfont}
\newcommand\textem[1]{%
  \ifdefstrequal{\f@series}{\bfdefault}
    {\@textemph{\CTEXunderline{#1}}}
    {\@textemph{#1}}%
}
\RenewDocumentCommand\emph{s o m}{%
  \IfBooleanTF{#1}
    {\textem{#3}}
    {\IfNoValueTF{#2}
      {\textem{#3}\index{#3}}
      {\textem{#3}\index{#2}}%
     }%
}
\makeatother   

% ====================================================================================
% 西北农林科技大学各单位名称
% ====================================================================================
\newcommand{\nwafu}{西北农林科技大学}
\newcommand{\cie}{信息工程学院}

% ====================================================================================
% 专用名词
% ====================================================================================
\newcommand{\teacharch}{教学档案}
\newcommand{\awardtitle}{\nwafu{}\teacharch{}系列\LaTeX{}}
\newcommand{\awardname}{\awardtitle{}模板的研发与推广}
\newcommand{\architems}{学位论文、试题试卷、演示文稿、教案、点名册}
\newcommand{\ltxcourse}{《\LaTeX{}排版技术》公共选修课}
\newcommand{\git}{分布式版本控制系统Git}
\newcommand{\github}{Github平台}

% ====================================================================================
% cquotes宏包的中文引号样式
% ====================================================================================
\DeclareQuoteStyle{zhquotestyle}% style name
    {\symbol{"201C}}% opening outer mark
    {\symbol{"201D}}% closing outer mark
    {\symbol{"2018}}% opening inner mark
    {\symbol{"2019}}% closing inner mark

\setquotestyle{zhquotestyle}

% ====================================================================================
% 改变表格字体
% ====================================================================================
\BeforeBeginEnvironment{tabular}{\small}%

%%% Local Variables:
%%% mode: latex
%%% TeX-master:"../main.tex"
%%% End:

% 专用术语宏命令进行必要的设置
% ====================================================================================
% 西北农林科技大学各单位名称
% ====================================================================================
\newcommand{\nwafu}{西北农林科技大学}
\newcommand{\cie}{信息工程学院}

% ============自定义专有名词命令============
\newcommand{\cl}{\texttt{C}语言}
\newcommand{\ccpp}{\texttt{C/C++}}
\newcommand{\win}{\texttt{Windows}}
\newcommand{\ide}{\texttt{IDE}}
\newcommand{\gcc}{\texttt{GCC}}
\newcommand{\gpp}{\texttt{G++}}
\newcommand{\gnu}{\texttt{GNU}}
\newcommand{\cb}{\texttt{Code::Blocks}}
\newcommand{\mgww}{\texttt{MinGW}}
\newcommand{\mgw}{\texttt{MinGW32}}
\newcommand{\mgwww}{\texttt{MinGW-w64}}
\newcommand{\lumos}{\texttt{Linux}、\texttt{Unix}、\texttt{Mac OS}}
\newcommand{\unix}{\texttt{UNIX}}
\newcommand{\lnx}{\texttt{Linux}}
\newcommand{\mk}{\texttt{make}}
\newcommand{\ph}{\texttt{Path}}
\newcommand{\cmdd}{\texttt{cmd}}
\newcommand{\gdb}{\texttt{gdb}调试器}
\newcommand{\vside}{\texttt{Visual Studio}}
\newcommand{\mfile}{\texttt{Makefile}}
\newcommand{\tgt}{\texttt{target}}
\newcommand{\prqt}{\texttt{prerequisites}}
\newcommand{\cbv}{\texttt{17.12}}
\newcommand{\db}{\texttt{DEBUG}}
\newcommand{\dbger}{\texttt{Debugger}}
\newcommand{\cdb}{\texttt{cdb}调试器}
\newcommand{\gdbcmd}{\texttt{(gdb)}}
\newcommand{\bug}{\texttt{BUG}}
\newcommand{\ieee}{\texttt{IEEE754}标准}
\newcommand{\ascii}{\texttt{ASCII}}
\newcommand\vararg{变长形参列表}
\newcommand\varargfun{\vararg{}函数}
\newcommand{\cg}{\texttt{CGraph2D} 图形库}
\newcommand{\git}{分布式版本控制系统\texttt{Git}}
\newcommand{\github}{\texttt{Github}平台}
% ====================================


%%% Local Variables:
%%% mode: latex
%%% TeX-master:"../main.tex"
%%% End:

% 乱数假文宏包
\usepackage{zhlipsum}

\title{\bfseries\sffamily 数据类型错误引起的死循环问题}
\author{\zihao{4} \fangsong 耿楠\\\small \songti 信息
  工程学院,陕西$\cdot$杨凌,712100}
\date{\today}

% 摘要内容
\begin{abstract}
  针对一个\cl{}程序设计陷入\enquote{死循环}的问题,采用中\db{}技术,通
  过单步跟踪和分析程序运行过程中的各个变量值的变化,定位了引起程序死循
  环错误代码,确定了变量类型是引发错误的原因,并提出了对应解决方案。实
  验表明,在浮点数计算时,将其结果存储到整型变量,会出现截断误差,使
  程序产生分支错误,进而会造成\enquote{死循环}。同时也可以得知,当程序
  出现错误,特别是出现逻辑错误时,使用\db{}跟踪和调试程序是非常有必要的。
\end{abstract}
% 关键词内容(用英文","分割)
\keywords{数据类型, 死循环, 截断误差, \db{}}

\begin{document} %在document环境中撰写文档
% 排版标题
\maketitle
\thispagestyle{empty}
% 排版摘要
\makeabstract

% 排版具体内容
\section{摘要}
摘要内容请置于\verb|abstract|环境中,关键词内容用英文\enquote{,}分割
后,置于\verb|\keywords{..., ..., ...}|命令中。

\verb|abstract|环境和\verb|\keywords{}|命可以在导言区,也可以在正文区。

设置好摘要内容和关键词内容后,在正文区用\verb|\maketitle|命令排版完题
目及作者、日期后,用\verb|\makeabstract|命令排版摘要。
\section{浮动体}
在\enquote{nwafucoursepaper.cls}模板中,引入了floatrow宏包进行浮动体排
版。有关该宏包的使用细节,请在命令行使用\enquote{texdoc floatrow}命令
查看其使用说明书。
\section{插图标注}
在\enquote{nwafucoursepaper.cls}模板中,引入了改自tikz-imagelabels宏包
的tikz-imglabels宏包,利用TiKZ为插图进行标注。该宏包的使用细节与
tikz-imagelabels完全一致,请在命令行使用\enquote{texdoc tikz-imagelabels}命令
查看其使用说明书。
\section{流程图}
在\enquote{nwafucoursepaper.cls}模板中,引入了自己开发的tikz-flowchart宏包
进行流程图的绘制。请在\github{}查看\href{https://github.com/registor/tikz-flowchart}{tikz-flowchart宏包}的使用说明书。
\section{标题和列表环境}
\subsection{二级标题}
\subsubsection{三级标题}
\zhlipsum[1]
\subsection{列表环境}
在\enquote{nwafucoursepaper.cls}模板中,基于enumitem宏包分别对\verb|itemize|、
\verb|enumerate|和\verb|description|三个环境的各个距离参数进行了修正,以使其排版结果
符合中文习惯的首先缩进格式。
\subsubsection{itemize环境}
\begin{itemize}
\item 床前明月光,床前明月光,床前明月光,床前明月光,床前明月光,床前明月光,床前明月光。
\item 疑是地上霜,疑是地上霜,疑是地上霜,疑是地上霜,疑是地上霜,疑是地上霜,疑是地上霜。
\item 举头望明月,举头望明月,举头望明月,举头望明月,举头望明月,举头望明月,举头望明月。
\item 低头思故乡,低头思故乡,低头思故乡,低头思故乡,低头思故乡,低头思故乡,低头思故乡。
\end{itemize}
\subsubsection{enumerate环境}
\begin{enumerate}
\item 床前明月光,床前明月光,床前明月光,床前明月光,床前明月光,床前明月光,床前明月光。
\item 疑是地上霜,疑是地上霜,疑是地上霜,疑是地上霜,疑是地上霜,疑是地上霜,疑是地上霜。
\item 举头望明月,举头望明月,举头望明月,举头望明月,举头望明月,举头望明月,举头望明月。
\item 低头思故乡,低头思故乡,低头思故乡,低头思故乡,低头思故乡,低头思故乡,低头思故乡。
\end{enumerate}
\subsubsection{description环境}
\begin{description}
\item[床前明月光],床前明月光,床前明月光,床前明月光,床前明月光,床前明月光,床前明月光。
\item[疑是地上霜],疑是地上霜,疑是地上霜,疑是地上霜,疑是地上霜,疑是地上霜,疑是地上霜。
\item[举头望明月],举头望明月,举头望明月,举头望明月,举头望明月,举头望明月,举头望明月。
\item[低头思故乡],低头思故乡,低头思故乡,低头思故乡,低头思故乡,低头思故乡,低头思故乡。
\end{description}

\section{\enquote{emph}强调字体}
在\enquote{nwafucoursepaper.cls}模板中,重定义强调字体,将默认强调字体
是italic,中文用楷体代替操作更换为加粗操作,用加粗后的字体表示强调。
\section{文本框盒子}
文本框盒子继承于自己开发的boxie宏包,其使用细节请在\github{}查看
\href{https://github.com/registor/boxiesty}{boxie宏包}的使用说明书。
同时,在该宏包的基础上,为boxie宏包添加加了摘自于
\href{https://github.com/WisdomFusion/latex-templates/tree/master/progartcn}{progartcn
  论文模板}的\enquote{标题}、\enquote{注意}、\enquote{重要}、
\enquote{技巧}和\enquote{警告}文本框环境代码\footnote{本节示例摘自于
  该模板中的tutorial-sample.tex文件}。
\subsection{\enquote{标题}文本框}
标题文本框环境的使用格式为:

\verb|\begin{titledBox}{<title>} <content> \end{titledBox}|

\begin{titledBox}{HTTP/Console 内核}
  HTTP 内核继承自 \verb|Illuminate\Foundation\Http\Kernel| 类,该类定义了一个 \verb|bootstrappers| 数组,这个数组中的类在请求被执行前运行,这些 \verb|bootstrappers| 配置了错误处理、日志、检测应用环境以及其它在请求被处理前需要执行的任务。
\end{titledBox}
\subsection{\enquote{注意}文本框}
注意文本框环境的使用格式为:

\verb|\begin{noteBox} <content> \end{noteBox}|

\begin{noteBox}
  HTTP 内核继承自 \verb|Illuminate\Foundation\Http\Kernel| 类,该类定义了一个 \verb|bootstrappers| 数组,这个数组中的类在请求被执行前运行,这些 \verb|bootstrappers| 配置了错误处理、日志、检测应用环境以及其它在请求被处理前需要执行的任务。
\end{noteBox}

\subsection{\enquote{重要}文本框}
重要文本框环境的使用格式为:

\verb|\begin{importantBox} <content> \end{importantBox}|

\begin{importantBox}
  HTTP 内核继承自 \verb|Illuminate\Foundation\Http\Kernel| 类,该类定义了一个 \verb|bootstrappers| 数组,这个数组中的类在请求被执行前运行,这些 \verb|bootstrappers| 配置了错误处理、日志、检测应用环境以及其它在请求被处理前需要执行的任务。
\end{importantBox}
\subsection{\enquote{技巧}文本框}
技巧文本框环境的使用格式为:

\verb|\begin{tipBox} <content> \end{tipBox}|

\begin{tipBox}
  HTTP 内核继承自 \verb|Illuminate\Foundation\Http\Kernel| 类,该类定义了一个 \verb|bootstrappers| 数组,这个数组中的类在请求被执行前运行,这些 \verb|bootstrappers| 配置了错误处理、日志、检测应用环境以及其它在请求被处理前需要执行的任务。
\end{tipBox}
\subsection{\enquote{警告}文本框}
警告文本框环境的使用格式为:

\verb|\begin{warningBox} <content> \end{warningBox}|

\begin{warningBox}
  HTTP 内核继承自 \verb|Illuminate\Foundation\Http\Kernel| 类,该类定义了一个 \verb|bootstrappers| 数组,这个数组中的类在请求被执行前运行,这些 \verb|bootstrappers| 配置了错误处理、日志、检测应用环境以及其它在请求被处理前需要执行的任务。
\end{warningBox}

\section{交叉引用}
在\enquote{nwafucoursepaper.cls}模板中,交叉引用基于cleveref宏包实现,
用\verb|\autoref|命令实现引用,并对图、表、节、小节、公式、代码等引用
标记字/词进行了设置。如对一人标签为\enquote{fig:01}的图使用
\verb|\autoref{fig:01}|便可以得到\enquote{图 XX}的结果。

\section{已载入的宏包}
在\enquote{nwafucoursepaper.cls}模板中,已引入的宏包有:
\verb|etoolbox|、\verb|enumitem|、\verb|amsmath|、\verb|mathrsfs|、
\verb|amsfonts|、\verb|booktabs|、\verb|colortbl|、\verb|multirow|、
\verb|makecell|、\verb|multicol|、\verb|ulem|、\verb|floatrow|、
\verb|wrapfig|、\verb|boxie|、\verb|tikz-imglabels|、
\verb|tikz-flowchart|、\verb|hyperref|、\verb|cleveref|、
\verb|bookmark|、\verb|graphicx|、\verb|geometry|、
\verb|environ|、\verb|fancyhdr|、\verb|zhlineskip|、\verb|caption|,
无需再次引入这些宏包。

\section{重要文件}
\begin{importantBox}
  在使用在\enquote{nwafucoursepaper.cls}模板前请确保:
  \verb|tikz-flowchart.sty|、\verb|boxie.sty|、
  \verb|tikz-flowchart.sty|、\verb|tikz-imglabels.sty|这四个文件在当前
  工作文件夹中。
\end{importantBox}


\end{document}

%%% Local Variables:
%%% mode: latex
%%% TeX-master: t
%%% End:
